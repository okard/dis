\documentclass[a4paper]{article}
\usepackage[utf8]{inputenc}
\usepackage[T1]{fontenc}
\usepackage[english]{babel}
\usepackage{amsmath}
\usepackage{mathptmx}
\usepackage{multicol}
\usepackage{environ}

\usepackage{listings}
\usepackage[usenames,dvipsnames]{color} 

%%%%%%%%%%%%%%%%%%%%%%%%%%%%%%%%%%%%%%%%%%%%%%%%%%%%%%%%
%% Lang Spec
\lstdefinelanguage{Dis}
{morekeywords={package, import, class, trait, def, var, val, type, return, this, if, else, switch, case, for,
void, bool, int, uint, string},
sensitive=true,
morecomment=[l]{//},
morecomment=[s]{/*}{*/},
morestring=[b]"
}
\lstset{%
  language=Dis,
  basicstyle=\tiny\ttfamily,
  commentstyle=\itshape\color{green},
  keywordstyle=\bfseries\color{blue},
  stringstyle=\color{red},
  showspaces=false,
  showtabs=false,
  columns=fixed,
  numbers=left,
  frame=single,
  numberstyle=\tiny,
  breaklines=true,
  showstringspaces=false,
  xleftmargin=1cm
}%

\NewEnviron{ebnf}{%
\begin{equation*}\begin{split}
  \BODY
\end{split}\end{equation*}
}
  
  
\title{Dis Language Specification}
\date{\today}
\author{}

%%%%%%%%%%%%%%%%%%%%%%%%%%%%%%%%%%%%%%%%%%%%%%%%%%%%%%%%%%%%%%%%%%%%%%%%%%%%%%%%%%
%% Start Document
\begin{document}
\maketitle
\begin{center}
Version 1.0
\end{center}
\newpage
\tableofcontents
\newpage

\section{Introduction}
This Document describes the Dis programming language, these document is licensed under Creative Commons Attribution-NoDerivs 3.0 Unported License.

\section{About}
The dis programming language is a multi paradigm programming language, designed with following paradigm in mind:
\begin{itemize}
\item Meta-programming
\item Object-oriented programming
\item Functional programming
\item Imperative programming
\item Modular
\end{itemize}

Other concepts:
\begin{itemize}
\item Type less, do more
\item Clear readable syntax
\item Programming language as tool, not as philosophy
\end{itemize}

%%%%%%%%%%%%%%%%%%%%%%%%%%%%%%%%%%%%%%%%%%%%%%%%%%%%%%%%%%%%%%%%%%%%%%%%%%%%%%%%%%
%% Lexical
\section{Lexical}
Basic:

\begin{itemize}
\item $Alpha \rightarrow [a-zA-Z\_]+$
\item $Number \rightarrow [0-9]+$
\item $Identifier \rightarrow Alpha (Alpha / Number)+$
\end{itemize}

A Identifier can be one of following keywords:
\begin{multicols}{4}
\begin{itemize}
\item package 
\item trait 
\item class 
\item def 
\item var 
\item val 
\item type 
\item for 
\item while 
\item do 
\item if 
\item switch 
\item as
\end{itemize}
\end{multicols}

A identifier can also be one of the following built in types
\subsection{Types}
Built-in:

\begin{multicols}{4}
\begin{itemize}
\item bool 
\item byte 
\item ubyte 
\item short 
\item ushort 
\item int 
\item uint 
\item long 
\item ulong 
\item float 
\item double 
\item ptr 
\item char 
\end{itemize}
\end{multicols}

User defined types:
Standard Library
\begin{itemize}
\item Classes, 
\item Functions, 
\item Traits, 
\item Structures
\item Enums
\item Delegates

\end{itemize}

%%%%%%%%%%%%%%%%%%%%%%%%%%%%%%%%%%%%%%%%%%%%%%%%%%%%%%%%%%%%%%%%%%%%%%%%%%%%%%%%%%
%% GrammarStandard Library
\section{Grammar}

\subsection{Declarations}

\subsubsection*{Packages}
\[PackageDeclaration \rightarrow "package" identifier;\]

\subsubsection*{Variables \& Values}
\begin{ebnf}
	VariableDecl & \rightarrow "var"\; identfifier\; [":" identifier] ["=" Literal] (";" | CRLF) \\
	ValueDecl & \rightarrow "val"\; identfifier\; [":" identifier] ["=" Literal] (";" | CRLF)
\end{ebnf}



var val
\lstinline!var i : int = 5;!

\subsubsection*{Functions}
def
\lstinline!def square(x) = x**2;!

calling convention

\subsubsection*{Classes}
class
\lstinline!class foo {}!
calling convention
constructor
destructor
operator overloading

\subsubsection*{Traits}
trait
\begin{lstlisting}
trait printable
{
	def print(): void;
}

class foo
{
	def print = io.writeln("foo");
}

//class foo can be printable
(foo() as printable).print();
\end{lstlisting}

\subsubsection*{Enums, Structure, Delegates, Alias, Variant}
The type keyword is used to declare utility types, a dis program isn't based on these types but can use them as utility.

\begin{lstlisting}
type myInteger int
type point { x: int; y : int; }
type myBoolean = {True, False}
type dgTest (int) => bool
type Number = byte | short | int | long | float | double;
\end{lstlisting}


\subsection{Statements}
for while do (mixins) try catch finally

\subsection{Expression}
\begin{itemize}
\item BinaryOperator: \& | + - * ** / \% \textasciicircum{} 
\item LogicalOpVersion 1.0erator: \&\&  ||
\end{itemize}
if 
switch

%%%%%%%%%%%%%%%%%%%%%%%%%%%%%%%%%%%%%%%%%%%%%%%%%%%%%%%%%%%%%%%%%%%%%%%%%%%%%%%%%%
%% Semantic
\section{Semantic}
Default Runtime and Standard Library Imports
Operator Overloading

%%%%%%%%%%%%%%%%%%%%%%%%%%%%%%%%%%%%%%%%%%%%%%%%%%%%%%%%%%%%%%%%%%%%%%%%%%%%%%%%%%
%% Name Mangarationarationling
\section{Name Mangeling}

%%%%%%%%%%%%%%%%%%%%%%%%%%%%%%%%%%%%%%%%%%%%%%%%%%%%%%%%%%%%%%%%%%%%%%%%%%%%%%%%%%
%% Runtime
\section{Runtime}

%%%%%%%%%%%%%%%%%%%%%%%%%%%%%%%%%%%%%%%%%%%%%%%%%%%%%%%%%%%%%%%%%%%%%%%%%%%%%%%%%%
%% Standard Library
\section{Standard Library}
std.io;
std.string;

%%%%%%%%%%%%%%%%%%%%%%%%%%%%%%%%%%%%%%%%%%%%%%%%%%%%%%%%%%%%%%%%%%%%%%%%%%%%%%%%%%
%% Standard Library
\section{Future}
Domain Specific Language


\end{document} 
