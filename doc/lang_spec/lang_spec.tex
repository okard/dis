\documentclass[a4paper,fleqn]{article}

\usepackage[utf8]{inputenc}
\usepackage[T1]{fontenc}
\usepackage[english]{babel}
\usepackage{amsmath}
\usepackage{mathptmx}
\usepackage{multicol}
\usepackage{environ}

\usepackage{listings}
\usepackage[usenames,dvipsnames]{color} 

%%%%%%%%%%%%%%%%%%%%%%%%%%%%%%%%%%%%%%%%%%%%%%%%%%%%%%%%
%% Lang Spec
\lstdefinelanguage{Dis}
{morekeywords={package, import, class, trait, def, var, val, type, return, this, if, else, switch, case, for,
void, bool, int, uint, string},
sensitive=true,
morecomment=[l]{//},
morecomment=[s]{/*}{*/},
morestring=[b]"
}
\lstset{%
  language=Dis,
  basicstyle=\tiny\ttfamily,
  commentstyle=\itshape\color{green},
  keywordstyle=\bfseries\color{blue},
  stringstyle=\color{red},
  showspaces=false,
  showtabs=false,
  columns=fixed,
  numbers=left,
  frame=single,
  numberstyle=\tiny,
  breaklines=true,
  showstringspaces=false,
  xleftmargin=1cm
}%

\NewEnviron{ebnf}{%
\begin{equation*}\begin{split}
  \BODY
\end{split}\end{equation*}
} 

\title{Dis Language Specification}
\date{\today}
\author{}

%%%%%%%%%%%%%%%%%%%%%%%%%%%%%%%%%%%%%%%%%%%%%%%%%%%%%%%%%%%%%%%%%%%%%%%%%%%%%%%%%%
%% Start Document
\begin{document}
\maketitle
\begin{center}
Version X.X DRAFT
\end{center}

\newpage
\tableofcontents
\newpage

\section*{License}

\begin{flushleft}
This Document describes the Dis programming language, these document is licensed under Creative Commons Attribution-NoDerivs 3.0 Unported License.

\url{http://creativecommons.org/licenses/by-nd/3.0/}
\end{flushleft}

\section{Introduction}

The dis programming language is a multi paradigm programming language, designed with following paradigm in mind:
\begin{itemize}
\item Meta-programming
\item Object-oriented programming
\item Functional programming
\item Imperative programming
\item Modular
\end{itemize}

Other concepts:
\begin{itemize}
\item Type less, do more
\item Clear readable syntax
\item Programming language as tool, not as philosophy
\end{itemize}

%%%%%%%%%%%%%%%%%%%%%%%%%%%%%%%%%%%%%%%%%%%%%%%%%%%%%%%%%%%%%%%%%%%%%%%%%%%%%%%%%%
%% Lexical
\section{Lexical}
Basic:

\begin{itemize}
\item $Alpha \rightarrow [a-zA-Z\_]+$
\item $Number \rightarrow [0-9]+$
\item $HexNumber \rightarrow 0x[0-9a-f][0-9a-f]$
\item $Identifier \rightarrow Alpha (Alpha / Number)+$
\end{itemize}

A Identifier can be one of following keywords:
\begin{multicols}{4}
\begin{list}{}{}
\item package 
\item import
\item trait 
\item class 
\item def 
\item var 
\item val 
\item type 
\item for 
\item while 
\item do 
\item if 
\item switch 
\item as
\item in
\item this
\item true
\item false
\item null
\end{list}
\end{multicols}

Symbols \& Operator
\begin{multicols}{5}
\begin{list}{}{}
\item +
\item -
\item *
\item /
\item \%
\item **

\item <
\item >

\item =
\item !
\item ==
\item <=
\item >=
\item +=
\item *=
\item /=

\item ++
\item --
\item ::


\item \&
\item \&\&
\item |
\item ||
\item \^{}

\item :=
\item <-
\item ->

\item \textasciitilde{}

\item \{
\item \}
\item (
\item )
\item {[}
\item ]
\item .
\item :
\item ;
\end{list}
\end{multicols}



%%%%%%%%%%%%%%%%%%%%%%%%%%%%%%%%%%%%%%%%%%%%%%%%%%%%%%%%%%%%%%%%%%%%%%%%%%%%%%%%%%
%% Grammar
%%%%%%%%%%%%%%%%%%%%%%%%%%%%%%%%%%%%%%%%%%%%%%%%%%%%%%%%%%%%%%%%%%%%%%%%%%%%%%%%%%
%% Syntax/Grammar
\section{Syntax / Grammar}

 
\subsection{Types}
Built-in:
\begin{multicols}{4}
\begin{itemize}
\item bool 
\item byte 
\item ubyte 
\item short 
\item ushort 
\item int 
\item uint 
\item long 
\item ulong 
\item float 
\item double 
\item ptr 
\item char 
\end{itemize}
\end{multicols}

User defined types:
\begin{itemize}
\item Classes, 
\item Functions, 
\item Traits, 
\item Structures
\item Enums
\item Delegates
\item Alias
\item Variants
\end{itemize}

\subsection{Declarations}

\subsubsection{Packages}

\begin{ebnf}
PackageDeclaration & \rightarrow "package" identifier\; \\
\end{ebnf}

\subsubsection{Types}

\begin{itemize}
\item Identifier
\item Identifier.Identifier
\item Identifier[]
\item Identifier!Identifier
\item Identifier!(Identifier, Identifier)
\item {[}Identifier]
\item Identifier *
\item ref Type
\item ptr Type
\end{itemize}

%% Variables %%%%%%%%%%%%%%%%%%%%%%%%%%%%%%%%%%%%%%%%%%%%%%%%%%%%%
\subsubsection{Variables \& Values}
\begin{ebnf}
	VariableDecl & \rightarrow "var"\; Identfifier\; [":"\; TypeIdentifier] ["=" Literal] (";" | CRLF) \\
	ValueDecl & \rightarrow "val"\; Identfifier\; [":"\; TypeIdentifier] ["=" Literal] (";" | CRLF)
\end{ebnf}

\begin{lstlisting}
// Explicit Typing
var i : int = 5;
val f : float = 5.0;
// Implicit Typing
var i = 5;
var b = 0x0;

\end{lstlisting}

\subsubsection{Constants}
\begin{ebnf}
	ConstantDecl & \rightarrow "const"\; Identfifier\; [":"\; TypeIdentifier] ["=" Literal] (";" | CRLF) \\
\end{ebnf}

%% Functions %%%%%%%%%%%%%%%%%%%%%%%%%%%%%%%%%%%%%%%%%%%%%%%%%%%%%
\subsubsection{Functions}
\begin{ebnf}
CallingConv & \rightarrow "("\; Identifier\; ")" \\
FunctionDeclaration & \rightarrow "def"\; [CallingConv] Identifier\; Identifier\; ["("\; ")" ]\; \\
\end{ebnf}

% Example
\begin{lstlisting}
def main
{
}

def square(x) = x**2;
\end{lstlisting}

calling convention

%% Classes %%%%%%%%%%%%%%%%%%%%%%%%%%%%%%%%%%%%%%%%%%%%%%%%%%%%%
\subsubsection{Classes}
class
\lstinline!class foo {}!
calling convention
constructor
destructor
operator overloading

%% Traits %%%%%%%%%%%%%%%%%%%%%%%%%%%%%%%%%%%%%%%%%%%%%%%%%%%%%
\subsubsection{Traits}
trait
\begin{lstlisting}
trait printable
{
	def print(): void;
}

class foo
{
	def print = io.writeln("foo");
}

//class foo can be printable
(foo() as printable).print();
\end{lstlisting}

%% Types Decl %%%%%%%%%%%%%%%%%%%%%%%%%%%%%%%%%%%%%%%%%%%%%%%%%%%%%
\subsubsection{Enums, Delegates, Alias, Variant}
The type keyword is used to declare utility types, a dis program isn't based on these types but can use them as utility.

\begin{lstlisting}
type myInteger int
type myBoolean = {True, False}
type dgTest def(:int) : bool
type Number = byte | short | int | long | float | double;
\end{lstlisting}


\subsection{Statements}
\subsubsection{for-Statement}
\subsubsection{while-do-Statement}
\subsubsection{try-catch-finally-Statement}
\subsubsection{return-Statement}
\begin{ebnf}
ReturnStatement & \rightarrow "return" Expression \\
\end{ebnf}

\subsubsection{Expression-Statement}
\begin{ebnf}
BlockStatement & \rightarrow {Statement} \\
\end{ebnf}

block

(mixins)

\subsection{Expression}


\subsubsection{Binary-Expression}
\begin{itemize}
\item BinaryOperator: \& | + - * ** / \% \textasciicircum{} 
\item LogicalOperator: \&\&  ||
\end{itemize}

\subsubsection{Unary-Expression}

\subsubsection{if-Expression}
\subsubsection{switch-case-Expression}
\subsubsection{cast-Expression}
\subsubsection{lambda-Expression}
\subsubsection{if-Expression}




%%%%%%%%%%%%%%%%%%%%%%%%%%%%%%%%%%%%%%%%%%%%%%%%%%%%%%%%%%%%%%%%%%%%%%%%%%%%%%%%%%
%% Semantic
%%%%%%%%%%%%%%%%%%%%%%%%%%%%%%%%%%%%%%%%%%%%%%%%%%%%%%%%%%%%%%%%%%%%%%%%%%%%%%%%%%
%% Semantic

\section{Semantic}

% Packages
% Declarations
% Statements
% Expressions

% Special: main function
% Libraries main function?

\begin{itemize}
\item Implicit Typing
\item Default Runtime and Standard Library Imports
\item Operator Overloading
\item implicit casts
\item All objects come from object
\item scope
\end{itemize}



%%%%%%%%%%%%%%%%%%%%%%%%%%%%%%%%%%%%%%%%%%%%%%%%%%%%%%%%%%%%%%%%%%%%%%%%%%%%%%%%%%
%% ABI & Runtime
%%%%%%%%%%%%%%%%%%%%%%%%%%%%%%%%%%%%%%%%%%%%%%%%%%%%%%%%%%%%%%%%%%%%%%%%%%%%%%%%%%
%% ABI & Runtime
\section{ABI \& Runtime}

A standard dis executable or library should be linked with the standard c library / runtime (libC) and the dis runtime (disrt).

\subsection{Types}
\begin{tabular}{|l|c|}
\hline 
bool    & • \\ \hline 
byte    & 8 Bit \\ \hline
ubyte   & 8 Bit \\ \hline
short   & 16 Bit \\ \hline
ushort  & 16 Bit \\ \hline
int     & 32 Bit \\ \hline
uint    & 32 Bit \\ \hline
long    & 64 Bit \\ \hline
ulong   & 64 Bit \\ \hline
float   & 32 Bit \\ \hline
double  & 64 Bit \\ \hline
ptr     & Platform-Specific \\ \hline
char    & 8/16/24/32 Bit UTF 8 Char \\ \hline
\end{tabular} 

 

%%%%%%%%%%%%%%%%%%%%%%%%%%%%%%%%%%%%%%%%%%%%%%%%%%%%%%%%%%%%%%%%%%%%%%%%%%%%%%%%%%
%% Name Mangling
\subsection{Name Mangeling}

%%%%%%%%%%%%%%%%%%%%%%%%%%%%%%%%%%%%%%%%%%%%%%%%%%%%%%%%%%%%%%%%%%%%%%%%%%%%%%%%%%
%% RT ABI Interface
\subsection{Runtime ABI Interface}

%%%%%%%%%%%%%%%%%%%%%%%%%%%%%%%%%%%%%%%%%%%%%%%%%%%%%%%%%%%%%%%%%%%%%%%%%%%%%%%%%%
%% Standard Library
%%%%%%%%%%%%%%%%%%%%%%%%%%%%%%%%%%%%%%%%%%%%%%%%%%%%%%%%%%%%%%%%%%%%%%%%%%%%%%%%%%
%% Standard Library
\section{Standard Library}
std.io;
std.string;

%%%%%%%%%%%%%%%%%%%%%%%%%%%%%%%%%%%%%%%%%%%%%%%%%%%%%%%%%%%%%%%%%%%%%%%%%%%%%%%%%%
%% Standard Library
\section{Future}
Embedded Domain Specific Language

\end{document} 
